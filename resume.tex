\documentclass{article}
\usepackage{scimisc-cv}
\usepackage{hyperref}
\hypersetup{
    colorlinks=true,
    linkcolor=blue,
    filecolor=magenta,
    urlcolor=blue,
}

\title{Scismic's Recommended CV for Biotech and Pharma Positions}
\author{Scismic: The Talent Matching Platform for The Life Sciences (www.scismic.com)}
\date{May 2020}

%% These are custom commands defined in scimisc-cv.sty
\cvname{Harsh P. Patel}
\cvpersonalinfo{
harshpatelfc22@gmail.com \cvinfosep
+1 (902)9891532 \cvinfosep
\href{https://harshp.tech/}{harshp.tech/} \cvinfosep
\href{https://github.com/Harshpatel249}{GitHub} \cvinfosep
\href{https://www.linkedin.com/in/harsh-patel-696a82176/}{\textcolor{blue}{LinkedIn}}
}

\begin{document}

% \maketitle %% This is LaTeX's default title constructed from \title,\author,\date

\makecvtitle %% This is a custom command constructing the CV title from \cvname, \cvpersonalinfo

\section{Education}
\begin{itemize}

\item
\cvsubsection{Dalhousie University}[Expected June 2024]
\\\textit{Masters of Applied Computer Science}
\item
\cvsubsection{School of Engineering and Applied Science, Ahmedabad University}[June 2022]
\\\textit{Bachelor of Technology in Information and Communication Technology}
\\
CGPA: 3.1

\end{itemize}
 
\section{Technical Skills}

\begin{itemize}
\item \textbf{Programming Languages:} C, C++, C\#, JAVA, Python, JavaScript, Go, LaTex
\item \textbf{Databases:} SQL, NoSQL, PLSQL
\item \textbf{Web-dev stack:} MongoDB, ExpressJS, ReactJS, NodeJS, Django, Flutter (Dart), Tailwind CSS
\item \textbf{App-dev stack:} Flutter (Dart), Firebase (Flutterfire)
\item \textbf{Development tools/Frameworks:} Unity 3D, Löve 2D, metasploit, SUMO, GEMV$^2$, Git, Tensorflow
\item \textbf{Cloud technologies:} AWS, Firebase

\end{itemize}
 
\section{Projects}

%% Another custom command provide by scimisc-cv.sty.
%% First two argumetns are typeset on the first line in bold; 3rd and 4th arguments are typset on second line in italics. 2nd, 3rd and 4th arguments are OPTIONAL

\cvsubsection{
\href{https://github.com/Harshpatel249/CorVito}{CorVito - Movie recommendation/watchlist website [Team of 4]}
}[]
\item
\begin{itemize}
\item An interactive website where the user can make and maintain different lists like watched list, currently watching and wishlist.
\item The user can rate the already watched movies according to his liking and the website uses this data to feed into a ML model. This model then recommends other movies based on the data.
\item Aim of the project is to learn to implement ML without using python libraries and also make a website based on the principles of Human-Computer Interaction.
\end{itemize}
\textbf{Project tags:} 
Front-end: HTML, CSS, ReactJS, JavaScript, Human Computer Interaction | Machine learning: python, NumPy, Scikit-Learn, SciPy, MatPlotlib 

\cvsubsection{
\href{https://github.com/Harshpatel249/Echos}{Echos: American Sign Language Tutor and ASL to Text translator [Led a team of 8]}
}[]
\item
\begin{itemize}
\item An app like Duolingo which teaches the user Sign Language(ASL), it also includes a Sign-Language to Text conversion feature.
\item They can also use a feature which uses the camera and detects the gestures of a person and converts it to text. 
\item The software was built using the principles of DevOps software development.
\end{itemize}
\textbf{Project tags:} 
Front-end: Android Studio, Flutter (Dart) | Back-end: Firebase | Machine Learning: NumPy, Scikit-Learn, SciPy, MatPlotlib, TensorFlow, Keras


\cvsubsection{
\href{https://github.com/Harshpatel249/magnar2d}{Magnar - JAVA based 2D Platformer game [Team of 2]}}[]
\item
\begin{itemize}
\item A modern version of the Mario Bros where players can create their own levels by editing an image.   
\item The game does basic image processing and renders the level based on the processed image.

\end{itemize}
\textbf{Project tags:} JAVA, image processing, 2D platformer game
%% An example of leaving an argument empty

\section{Work Experience}

\cvsubsection{Full-Stack Developer Intern at Quicko (6 Months)(Dec'21 to June'22)}[]
\begin{itemize}
    \item This internship provided me the opportunity of developing the back-end and the Mobile and Desktop Progressive Web Apps for their product "MEET".\\
\textbf{Tech-Stack:} Flutter, Node.js, AWS DynamoDB, AWS lambdas 

\end{itemize}

\cvsubsection{Android Development Internship (3 Months)(May'21 to July'21)}[]
\begin{itemize}
    \item The internship included developing a full-stack android application with a responsive UI from scratch.\\
\textbf{Tech-Stack:} 
Front-end: Android Studio, Flutter (Dart) | Back-end: Firebase
\end{itemize}
\cvsubsection{Driver-Pedestrian Interaction Research Internship (2 Months)(June'21 to July'21)}[]
\begin{itemize}
    \item The internship included intensive research for developing a data-set for Driver-Pedestrian Interaction in Autonomous Driving.\\
\textbf{Tags:} Deep Learning, Machine Learning, Research

\end{itemize}

\cvsubsection{FullStack Web Development Internship (3 Months)(Aug'21 to October'21)}[]
\begin{itemize}
    \item The internship included developing a full stack website for the company GrowthCentral VC.\\
\textbf{Tech-Stack:} MongoDB, Express.js, ReactJS, Node.js

\end{itemize}
\cvsubsection{Teaching Assistant (Aug'21 to Dec'21)}[]
\begin{itemize}
    \item Teaching assistant at Ahmedabad University for the fall semester of 2021.
\end{itemize}

\section{Other Skills/Interests}
\begin{description}[widest=Langauges]
\item[Software]     Adobe premier pro
\item[Interests] Competitive Programming, portrait sketching, video games, guitar, volleyball
\end{description}


\end{document}