\documentclass{article}
\usepackage{scimisc-cv}
\usepackage{hyperref}
\hypersetup{
    colorlinks=true,
    linkcolor=blue,
    filecolor=magenta,
    urlcolor=blue,
}

\title{Scismic's Recommended CV for Biotech and Pharma Positions}
\author{Scismic: The Talent Matching Platform for The Life Sciences (www.scismic.com)}
\date{May 2020}

%% These are custom commands defined in scimisc-cv.sty
\cvname{Harsh Pranav Patel}
\cvpersonalinfo{
harshpatelfc22@gmail.com \cvinfosep
+1 (902)9891532 \cvinfosep
\href{https://harshp.tech/}{harshp.tech/} \cvinfosep
\href{https://github.com/Harshpatel249}{GitHub} \cvinfosep
\href{https://www.linkedin.com/in/harsh-patel-696a82176/}{\textcolor{blue}{LinkedIn}}
}

\begin{document}

% \maketitle %% This is LaTeX's default title constructed from \title,\author,\date

\makecvtitle %% This is a custom command constructing the CV title from \cvname, \cvpersonalinfo

\section{Education}
\begin{itemize}

\item
\cvsubsection{Dalhousie University}[Expected June 2024]
\\\textit{Masters of Applied Computer Science}
\item
\cvsubsection{School of Engineering and Applied Science, Ahmedabad University}[June 2022]
\\\textit{Bachelor of Technology in Information and Communication Technology}
\\
CGPA: 3.1

\end{itemize}
 
\section{Technical Skills}

\begin{itemize}
\item \textbf{Programming Languages:} C, C++, C\#, HTML5, CSS3, JAVA, Python, Dart, JavaScript (ES5/ES6), TypeScript, Go, LaTex
\item \textbf{Databases:} SQL, NoSQL, PLSQL
\item \textbf{Web-dev stack:} ExpressJS, ReactJS, Angular JS, NodeJS, Django, Flutter (Dart), Tailwind CSS
\item \textbf{App-dev stack:} Flutter (Dart), Firebase (Flutterfire)
\item \textbf{Development tools/Frameworks/CI-CD:} Unity 3D, Git, GitHub actions, graphQL, Postman, RESTful APIs, SASS, Linux, Netlify, Heroku, Docker
\item \textbf{Cloud technologies:} AWS, Firebase

\end{itemize}

\section{Work Experience}

\cvsubsection{Full-Stack Developer Intern \\ Quicko Infosoft Pvt Ltd (6 Months)(Dec'21 to June'22)}[]
\begin{itemize}
    \item Developed 3 different Progressive Web Apps as the lead developer for the product "MEET".
    \item Increased the user conversion rate by 65\% by ideating and implementing a new workflow with improved funnels.
    \item Decreased the order life-cycle from a maximum 1 week long duration to 1 hour per order by developing a new and revised order workflow.
    \item Developed and integrated over 100 API calls with 3 different UI workflows.\\
\textbf{Tech-Stack:} Flutter, NodeJS, AngularJS, JAVA services, AWS 

\end{itemize}

% \cvsubsection{Teaching Assistant  \\ Ahmedabad University (Aug'21 to Dec'21)}[]
% \begin{itemize}
%     \item Helped a batch of 120 students with their Embedded Systems Design course as a Teaching Assistant/Marker at Ahmedabad University.
%     \item Conducted the laboratory sessions for the course as the primary instructor and helped the students with implementing their lab assignments.
%     \item Assisted the students in ideating and building their project.
% \end{itemize}

% \cvsubsection{FullStack Web Developer Intern \\ Growth Central VC (3 Months)(Aug'21 to October'21)}[]
% \begin{itemize}
%     \item \\
% \textbf{Tech-Stack:} MongoDB, Express.js, ReactJS, Node.js

% \end{itemize}

\cvsubsection{Flutter Developer Intern \\ Ahmedabad University (3 Months)(May'21 to July'21)}[]
\begin{itemize}
    \item Created an application with the features identified by 12 months long research on mental health.
    \item Research observed that intentionally laughing 2-3 times a day improved the individual's mental health greatly.
    \item Application kept track of user's daily sessions, sent reminders and provided laugh along videos from the community.\\
\textbf{Tech-Stack:} 
Front-end: Android Studio, Flutter (Dart) | Back-end: Firebase
\end{itemize}

% \cvsubsection{Driver-Pedestrian Interaction Research Internship \\ Ahmedabad University (2 Months)(June'21 to July'21)}[]
% \begin{itemize}
%     \item Created a data-set from over 500 hours of dash-cam footage captured in the city of Ahmedabad, India for using in Intelligent Transportation Systems.
%     \item Extracted over 600 meaningful interactions between the driver and a pedestrian to include into the data-set.
%     \item The data-set was annotated with 14 different attributes for documenting driver-pedestrian interactions. \\
% \textbf{Tags:} Research, Deep Learning

% \end{itemize}



 
\section{Projects}

%% Another custom command provide by scimisc-cv.sty.
%% First two argumetns are typeset on the first line in bold; 3rd and 4th arguments are typset on second line in italics. 2nd, 3rd and 4th arguments are OPTIONAL

\cvsubsection{
\href{https://github.com/Harshpatel249/CorVito}{CorVito - Movie recommendation/watchlist website [Team of 4]}
}[]
\begin{itemize}
\item Developed an interactive website and movie recommendation ML model using Human-Computer Interaction principles and Gaussian Mixture Models. 
\item Used ReactJS with Material UI components to develop user-friendly design and workflow.
\item Users add their already watched movies into the watchlist. This data is then fed into the machine learning model for clustering and generating movie recommendations.
\end{itemize}
\textbf{Project tags:} 
Front-end: ReactJS, JavaScript, Material UI, Human Computer Interaction | Machine learning: python, NumPy, Scikit-Learn, SciPy, MatPlotlib 

\cvsubsection{
\href{https://github.com/Harshpatel249/Echos}{Echos: American Sign Language Tutor and ASL to Text translator [Led a team of 8]}
}[]
\begin{itemize}
\item An app like Duolingo with more than 25 video tutorials and related quiz content inside the app to teach the users American Sign Language (ASL).
\item A bonus feature that uses camera feed and deep learning model to convert all the 26 alphabets from ASL to English text.
\item Agile methodology was used for the development with 2 week long sprints and DevOps principles for pipelining the software development lifecycle.
\end{itemize}
\textbf{Project tags:} 
Front-end: Flutter (Dart) | Back-end: Firebase | Machine Learning: NumPy, Scikit-Learn, SciPy, MatPlotlib, TensorFlow, Keras


\cvsubsection{
\href{https://github.com/Harshpatel249/magnar2d}{Magnar - JAVA based 2D Platformer game [Team of 2]}}[]
\item
\begin{itemize}
\item 2-D platformer game built using JAVA inspired from Super Mario bros and Trap Adventure.
\item Used open-source sprite sheets for character animations and textures of game objects.
\item Pixel=based image processing was used for building the level using more than 8 different game objects and corresponding color codes.

\end{itemize}
\textbf{Project tags:} JAVA, image processing, 2D platformer game
%% An example of leaving an argument empty

% \section{Other Skills/Interests}
% \begin{description}[widest=Langauges]
% \item[Software]     Adobe premier pro
% \item[Interests] Competitive Programming, portrait sketching, video games, guitar, volleyball
% \end{description}


\end{document}